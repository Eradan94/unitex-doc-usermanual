\chapter{序列控制器}
\label{chap-sequence-automaton}
\index{序列控制器}

当地的语法建设可能是一个漫长的过程,该过程中语言重复多次相同的动作 \verb+Seq2Grf+该程序的目的是自动迅速生成当地语法

\bigskip
\noindent
这个程序可以通过线上命令或者点击‘文本’菜单上的‘构建自动序列’
这部分是命令 \verb+Seq2Grf+ 的使用说明~\ref{Seq2Grf}.

\bigskip
\noindent 对于给定的文件 (\verb+TEILite+ 或是文件格式 \verb+txt+ 或 \verb+SNT+ 当它们被预先标记过 \verb+{STOP}+) 这个程序构建一个独特的自动装置,该装置能识别文件中的所有序列

\bigskip
\noindent 我们需要特别注意该序列表的构建需要被图表识别

\bigskip
\noindent 本章介绍该程序支持的文件格式 \verb+Seq2Grf+, 构建自动序列表和使用通配符
\bigskip

\section{语料库序列}
\index{语料库序列}
我们所说的 \textit{语料库序列}\index{语料库序列} 或 \textit{合格的语料库}
\index{合格的语料库} 一个或多个字的序列表,我们想通过单一图画表示的当地语法来认识序列表
\bigskip
文库序列储存在一个文件中,该文件可能是以下格式 :
\begin{itemize}
\item 纯文本文件, 该文件中的序列被末端限定
\item 文件 \verb+SNT+ 已经预先处理 菜单显示: 序列已被限定
		由 \verb+{STOP}+
\item 文件 \verb+TEILite+ 文件中的序列被标签限定 \verb+xml+ 该
		格式 :
%不要删除下行

		\verb+{<seg type="sequence">example</seg>}+
\end{itemize}
\pagebreak 
\indent 因为文集包含了特别的序列,它必须由手工完成
这意味着你应该在纯文本文件中写下所有序列,并用换行分隔它们 (figure~\ref{fig8-1CorpusTxt}), 插入特定标识符 \verb+XML+

在一个现存文件中 \verb+TEILite+ (figure~\ref{fig8-3CorpusTEI}). 预处理文件 \verb+TXT+ ou \verb+XML+ 产生 \verb+SNT+ 被用于顺序结构的文件 (figure~\ref{fig8-2CorpusSNT}). 
该文件能作为入口使用
该产品图只能认识出正确划分的序列
当地的语法只是自动地从明确定义的序列语料库中产生
如果你有这样一个语库,那么节省的时间是相当可观的

\begin{figure}[h!]
	\begin{minipage}[h!]{0.5\linewidth}
		\centering
		\includegraphics[scale=0.6]{resources/img/fig8-1tomorrow.png}
		\caption{TXT\label{fig8-1CorpusTxt}}
		\label{fig7-TXT}
	\end{minipage}
	\hspace{0.1cm}
	\begin{minipage}[h!]{0.5\linewidth}
		\centering
		\includegraphics[scale=0.6]{resources/img/fig8-2tomorrowSNT.png}
		\caption{SNT\label{fig8-2CorpusSNT}}
	\end{minipage}
	\hspace{0.1cm}
\end{figure}
\begin{figure}[h!]
	\begin{minipage}[h!]{\linewidth}
		\centering
			\includegraphics[width=14cm]{resources/img/fig8-3tomorrowTEI.png}
			\caption{TEILite\label{fig8-3CorpusTEI}}
	\end{minipage}
\end{figure}

%%%%%%%%%%%%%%%%%%%%%%%%%%%%%%%%%%%%%%%%%%%%%%%%%%%%%%%%%%%%%%%%%%%

\section{使用}

要创建一个顺序的自动控制器, 
点击“测试”菜单上的“顺序结构”

你将会看到一个窗口的出现 figure~\ref{fig8-4Menu1}.
此窗口允许你定义参数,以产生一个顺序自动控制器
你应该遵循三个步骤:
\begin{itemize}
\item 选择语料库序列 : 
它可以是一个文件,其格式是上一节中所描述的三种格式之一。
根据文件扩展名的文件格式自动检测。
\item设置特定的选项 :
“应用美化算法”将代替每个框,这样产生的图是体积更小,更容易阅读。
"精确匹配情况"将令牌字面括号中的图,以便它不能识别的标记具有相同的字母,但与壳体的差异。


你可以设置附加选项,以产生一个图使近似匹配:你可以修复通配符的数量用于生产从原来的语料序列的新序列,并选择适当的通配符。
使用通配符的所有细节都在这一章 
\ref{approximation}

\item 选择图片保存的路径
\end{itemize}
\medskip
\begin{figure}[h!]
	\begin{minipage}[h!]{0.5\linewidth}	
		\centering
			\includegraphics[width=8cm]{resources/img/fig8-4Menu1.png}
			\caption{Menu automate de séquences\label{fig8-4Menu1}}
	\end{minipage}	
	\hspace{0.3cm}
	\begin{minipage}[h!]{0.5\linewidth}	
		\centering
			\includegraphics[width=8cm]{resources/img/fig8-4Menu2.png}
			\caption{Menu options de l'automate de séquences\label{fig8-4Menu2}}
	\end{minipage}
\end{figure}

\bigskip
\noindent 你能看到 figures~\ref{fig8-5GRFnoBeautify} et ~\ref{fig8-6GRFBeautify} 没有通配符或者没有美化的图片


\begin{figure}[h!]
	\begin{minipage}[h!]{0.5\linewidth}
		\centering
		\includegraphics[width=8cm]{resources/img/fig8-5GRFnoBeautify.png}	
		\caption{Automate sans l'option "beautify"\label{fig8-5GRFnoBeautify}}
	\end{minipage}
	\hspace{0.1cm}
	\begin{minipage}[h!]{0.5\linewidth}
		\centering
		\includegraphics[width=8cm]{resources/img/fig8-6GRFBeautify.png}
		\caption{Automate avec l'option "beautify"\label{fig8-6GRFBeautify}}
	\end{minipage}
	\hspace{0.1cm}
\end{figure}
\pagebreak

%%%%%%%%%%%%%%%%%%%%%%%%%%%%%%%%%%%%%%%%%%%%%%%%%%%%%%%%%%%%%%%%%%%%%%%%%%%

\section{近似查询}
\label{approximation}

当你使用该程序所产生的图形做出一个文字“定位器”,
\verb+Seq2Grf+, 你只发现存在于原始序列的主体序列。
序列类似于原始语料库可以存在于文本和被忽略,因为它们不包括在本语料库。
这些序列应该包含在顺序控制器。
为了包括这些序列,必须应用三种通配符,并产生一个图表,识别语料的所有序列和新序列。
每一个通配符能申一个产生新的序列的操作。

\begin{itemize}
	\item 插入:对于每个序列,加入所有序列中的
	\verb+<TOKEN>+ 原始序列的两个词之间插入自动控制器
	\item 替换:对于每个序列,加入所有序列中的 $i$	用 \verb+<TOKEN>+ 来替换
	\item 删除:对于每个序列,加入所有序列中被删除的位置
\end{itemize}
所有这些操作的可应用数次到原始序列。此语法到文本的应用程序允许引入近似在寻求文本序列。

如果使用通配符,产品图遵循这些规则 :
\begin{itemize}
	\item 原序列和衍生的序列被包括在控制器中,
	\item 空序列或序列中只有通配符,将不会添加到该图中
		  (这样的序列可以通过缺失或短序列的置换来制备)
	\item 序列的头部和尾部不能插入通配符
	\item 每一个序列的令牌都能够被通配符代替
\end{itemize}

使用通配符生成的图表包含许多错误的序列
并且应该对照语料库 \verb+Locate+ 只保留相应的序列。
这些序列可以用来产生一个你希望保存的新图形。

Le graphe de la figure ~\ref{fig8-7GRF1replace} a été produit avec remplacement de 1 token et avec
l'option "beautifying" activée. (cf. figure ~\ref{fig8-2CorpusSNT})
\begin{figure}[h!]
	\begin{center}
		\includegraphics[width=8cm]{resources/img/fig8-7GRF1replace.png}
		\caption{Automate avec un remplacement permis\label{fig8-7GRF1replace}}
	\end{center}
\end{figure}



