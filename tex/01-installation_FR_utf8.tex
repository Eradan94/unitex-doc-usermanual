\chapter{Installation d'Unitex}
\label{chap-install}

Unitex est un système multi-plateformes capable de fonctionner aussi bien sous
Windows que sous Linux ou OS~X. Ce chapitre décrit l’installation et le lancement d’Unitex
pour chacun de ces systèmes. Il présente également les procédures d’ajout de nouvelles
langues et de désinstallation.

\section{Licences}
\label{section-licences}
\index{Licence!LGPL}\index{Licence!LGPL}
Unitex est un logiciel libre. Cela signifie que le code source des programmes est distribué avec le
logiciel, et que chacun peut le modifier et le redistribuer. Le code des programmes d’Unitex est
sous licence LGPL (\cite{LGPL}), à l’exception de~:

\begin{enumerate}
\item la bibliothèque de manipulation d’expressions régulières TRE de Ville Laurikari (\cite{TRE}),
qui est sous une licence du genre des licences BSD à 2 clauses~;

\item la bibliothèque \verb+wingetopt+ de Todd Miller et de la Fondation NetBSD, également sous une licence
 du genre des licences BSD à 2 clauses, plus permissive que la licence LPGL~;

\item l'analyseur syntaxique Xerces2 Java Parser, de l'Apache Software Foundation, sous licence Apache~;

\item  la bibliothèque LibYAML de Kirill Simonov, qui est sous licence MIT, également plus permissive que la licence LGPL~;

\item la bibliothèque SVNKit de TMate Software, sous licence TMate.
\end{enumerate}

\noindent La licence LGPL est plus permissive que la licence GPL, car elle permet d’utiliser du code LGPL dans
des logiciels non libres. Dans les deux cas, le logiciel peut librement être utilisé et distribué.


\bigskip
\noindent Toutes les ressources linguistiques distribuées avec Unitex sont soumises à la licence LGPLLR
\index{Licence!LGPLLR} (\cite{LGPLLR}).

\bigskip
\noindent Le texte complet des licences LGPL, BSD à 2 clauses, Apache, MIT, TMate et LGPLLR
se trouve dans les annexes à la fin de ce manuel.

\section{Environnement d’exécution Java}
Unitex est composé d’une interface graphique écrite en Java et de programmes externes
écrits en \textit{C/C\kern-.05em\raisebox{.5ex}{++}\kern-.1em}. Ce mélange de langages de 
programmation permet d’avoir une application rapide et portable sous différents systèmes d’exploitation.


\bigskip
\noindent Afin de pouvoir utiliser l’interface graphique, il faut préalablement installer
un environnement d’exécution, communément appelé machine virtuelle \index{Java!machine virtuelle} ou
JRE\index{JRE} (Java Runtime Environment\index{Java!Runtime Environment}\index{Java!JRE}).

\bigskip
\noindent Pour fonctionner en mode graphique, Unitex nécessite une version 1.7 (ou plus récente)
de Java. Si vous avez une version trop ancienne de Java, Unitex se bloquera après que vous
ayez choisi votre langue de travail.


\bigskip
\noindent Vous pouvez télécharger librement la machine virtuelle correspondant à votre 
système d’exploitation sur le site de Sun Microsystems (\cite{site-java}) à l’adresse suivante : 
\url{http://java.sun.com}.

\bigskip
\noindent Si vous travaillez sous Linux ou OS X, ou si vous
utilisez une version de Windows gérant des comptes personnels pour les utilisateurs, il vous
faudra demander à votre administrateur système d’installer Java.


\section{Programme d'installation}
\begin{samepage}
\index{Fichier!d'installation}
Le programme d'installation d'Unitex/GramLab peut être téléchargé depuis cette page~:

\begin{center}
{\tt\url{\UnitexURLLatestReleases}}
\end{center}
\end{samepage}

\subsection{Sous Windows}
\index{Installation!sous Windows}
Le nom du fichier téléchargé sera par exemple~:

\begin{flushleft}
{\tt \UnitexPackageWin{}}
{\tt \UnitexPackageWinSF{}}
\end{flushleft}

\noindent Ensuite, double-cliquez sur ce fichier et suivez les instructions (fig.~\ref{fig-installer}).
Il est recommandé de désinstaller
toute version existante avant d'en installer une nouvelle. Unitex/GramLab sera installé
dans un répertoire (dossier)\index{Dossier|see{Répertoire}} situé
de préférence dans le répertoire  \verb+Program Files+, et qui sera appelé dans ce manuel
le répertoire système Unitex.\index{Répertoire!système Unitex}

\begin{figure}[!ht]
\begin{center}
\includegraphics[width=13cm]{resources/img/installer.png}
\caption{Programme d'installation sous Windows\label{fig-installer}}
\end{center}
\end{figure}


\bigskip
\noindent Une fois l'installation terminée, une icône Unitex et une icône GramLab apparaissent
sur le bureau~: double-cliquez dessus pour lancer Unitex ou GramLab (voir~\ref{section-first-use}).
(Si le programme d'installation n'a pas créé ces icônes, ouvrez le répertoire Unitex système~: il contient
plusieurs sous-répertoires, dont un est \verb+App+.  Ce dernier répertoire contient deux fichiers
\verb+Unitex.jar+\index{Fichier!\verbc{Unitex.jar}} et \verb+GramLab.jar+.
 Ce sont les fichiers Java qui lancent les interfaces graphiques. Double-cliquez sur l'un d'entre eux
pour lancer Unitex ou GramLab (voir~\ref{section-first-use}).
Pour faciliter le lancement du programme, il est conseillé de créer des raccourcis vers ces fichiers sur le bureau.)

\bigskip
\noindent Si vous désirez installer Unitex sur une machine Windows multi-utilisateurs, il est
préférable de demander à votre administrateur de le faire. Si vous êtes le seul utilisateur de
votre machine, vous pouvez effectuer l’installation vous-même.

\bigskip
\begin{samepage}
\noindent Le programme d'installation sous Windows peut aussi être lancé en ligne de commande et dans ce cas
il accepte plusieurs paramètres optionnels. En voici quelques-uns~:

\begin{itemize}
\itemsep1pt\parskip0pt\parsep0pt
\item
  \texttt{/AllUsers} \hspace{.15in} Par défaut i'installation vaudra pour tous les utilisateurs\\
\item
  \texttt{/CurrentUser} \hspace{.15in} Par défaut i'installation vaudra pour l'utilisateur seulement\\
\item
  \texttt{/D C:\textbackslash{}path\textbackslash{}without quotes\textbackslash{}} \hspace{.15in}
  Spécifie le répertoire d'installation par défaut\\
\item
  \texttt{/NCRC} \hspace{.15in} Saute le contrôle de redondance cyclique\\
\item
  \texttt{/S} \hspace{.15in} Supprime toutes les questions
\end{itemize}
\end{samepage}

\noindent Sous Windows~7, on peut avoir des problèmes avec le fichier de configuration d'Unitex, car Unitex
essaie de le créer dans le sous-répertoire Unitex et Windows~7 le lui interdit.

\subsection{Sous GNU/Linux et OS~X}
\index{Installation!sous Linux}\index{Installation!sous OS X}
Le nom du fichier téléchargé sera par exemple~:

\begin{flushleft}
{\tt \UnitexPackageLinux{}}
{\tt \UnitexPackageLinuxSF{}}
\end{flushleft}

\noindent Donnez-lui les droits d'exécution, par exemple par~:

\begin{flushleft}
{\tt chmod a+x \UnitexPackageLinux{}}
\end{flushleft}

\noindent Le fichier \verb+.run+ est une archive qu'on décompresse en l'exécutant~: 

\begin{flushleft}
{\tt ./\UnitexPackageLinux{}}
\end{flushleft}

\begin{samepage}
\noindent Le fichier d'installation sous GNU/Linux et OS~X accepte plusieurs paramètres de ligne de commande optionnels. En voici quelques-uns~:

\begin{itemize}
\itemsep1pt\parskip0pt\parsep0pt
\item
  \texttt{-\/-confirm} \hspace{.15in} Demander avant de lancer le script d'installation \\
\item
  \texttt{-\/-quiet} \hspace{.15in} N'afficher que les messages d'erreur\\
\item
  \texttt{-\/-noexec} \hspace{.15in} Ne pas lancer le script d'installation \\
\item
  \texttt{-\/-target dir}  \hspace{.15in} Spécifier le répertoire d'installation par défaut
\end{itemize}
\end{samepage}

\section{Installation manuelle}
\begin{samepage}

On peut aussi installer Unitex/GramLab manuellement à l'aide du paquet de distribution. Téléchargez-le depuis cette page~:
\nopagebreak
\begin{center}
{\tt\url{\UnitexURLLatestReleases/source}}
\end{center}

\noindent Le nom du fichier téléchargé sera par exemple~:
\begin{flushleft}
{\tt \UnitexPackageSource{}}
% The following does not work for now:
%{\tt \UnitexPackageSource{}}\index{File!{\tt\UnitexPackageSource{}}}
% Anyway, this filename is too long for the index, it would overlap another item
\end{flushleft}

\end{samepage}

\noindent  Créez un répertoire que vous nommez par exemple {\tt Unitex\UnitexVersion{}},
de préférence dans le répertoire  \verb+Program Files+, et qui sera appelé dans ce manuel
le répertoire système Unitex.\index{Répertoire!système Unitex} Décompressez-y le paquet de distribution.

\bigskip
\noindent Si votre ordinateur est sous un des systèmes d'exploitation suivants, l'installation est terminée~:
Windows (32 bits ou 64 bits), GNU/Linux (i686 ou x86\_64) et OS~X (10.7+). (S'il fonctionne sous un autre système Unix, comme
FreeBSD, ou s'il a une autre architecture de processeur, comme ARM, allez dans le répertoire \verb+App/install+ et lancez~:

\begin{flushleft}
\verb+sh setup+
\end{flushleft}

\noindent Ce script vérifie si Java est installé, compile les sources C++, crée les répertoires personnels de travail
Unitex et GramLab et met en place des raccourcis sur le bureau\footnote{ Si vous voulez uniquement compiler les sources C++,
extrayez les fichiers du paquet de distribution, placez-vous dans le répertoire {\tt Src/C++/build} et lancez {\tt make install}.}.)

\bigskip
\noindent Une fois l'installation terminée, le répertoire système Unitex contient plusieurs
sous-répertoires dont un est \verb+App+.
\begin{itemize}
\item Sous Windows~: \verb+App+ contient des fichiers \verb+Unitex.jar+\index{File!\verbc{Unitex.jar}}
et \verb+GramLab.jar+. Ce sont les fichiers Java qui lancent les interfaces graphiques. Double-cliquez sur l'un
d'entre eux pour lancer Unitex ou GramLab (voir~\ref{section-first-use}). Pour faciliter le lancement du programme,
il est conseillé de créer des raccourcis vers ces fichiers sur le bureau.
\item Sous Linux ou OS~X: le répertoire \verb+App+ contient deux scripts shell \verb+Unitex+\index{File!\verbc{Unitex}}
et \verb+GramLab+. Lancez l'un d'entre eux pour démarrer Unitex or GramLab (voir~\ref{section-first-use}). Si vous
avez lancé le script d'installation, il a normalement fait apparaitre sur le bureau des raccourcis vers ces fichiers.
\end{itemize}

%Ce dernier répertoire contient un fichier nommé
%\verb+Unitex.jar+\index{Fichier!\verbc{Unitex.jar}}.
% Ce fichier est l’exécutable Java qui lance l’interface graphique. Il vous suffit de double-cliquer
%dessus pour lancer le programme.
%Pour faciliter le lancement du programme, il est conseillé de créer un raccourci vers ce fichier sur le bureau.


%Pour installer Unitex sous Linux et OS X, il est recommandé d’être administrateur système. Décompressez le fichier %\verb+Unitex3.1beta.zip+ dans un répertoire nommé
%\verb+Unitex+, au moyen de la commande suivante :


%\bigskip \noindent \verb$unzip Unitex3.1beta.zip -d Unitex$

%\bigskip
%\noindent Ce répertoire sera appelé dans ce manuel le répertoire système Unitex.\index{Répertoire!système Unitex}
%Placez-vous ensuite dans le répertoire \verb|Unitex/Src/C++/build|   , et lancez la compilation des
%programmes au moyen de la commande :


%\bigskip \verb+make install+

%\bigskip
%\noindent ou si avez un ordinateur 64 bits avec la commande :
 
%\bigskip \verb+make install 64BITS=yes+

%\bigskip
%\noindent Créez ensuite un alias sur le modèle suivant :

%\bigskip \verb$alias unitex='cd /..../Unitex/App/ ; java -jar Unitex.jar'$


%\section{Installation sous OS X}

%\label{section-macos-install}
%\noindent NOTE: ce court tutoriel va vous expliquer comment installer et exécuter Unitex sous Mac OS
%X. Vos questions, commentaires, suggestions,
%corrections sont plus que bienvenus.
%\noindent Contact: \url{cedrick.fairon@uclouvain.be}


%\bigskip
%\noindent Une version officielle Oracle de Java existe pour OS X 10.7.3 (Lion) et plus récent.
%	Voir section ``Informations et configuration minimale requise pour l'installation et l'utilisation d'Oracle Java sur Mac OS X'', à %\url{https://www.java.com/fr/download/faq/java_mac.xml}

	

%\bigskip
%\noindent Il existe une distribution Java d'Apple for OS X 10.7 and higher.
%	Voir \url{https://support.apple.com/kb/DL1572}. Pour OS X 10.6, il existe une autre distribution Apple sur %\url{https://support.apple.com/kb/DL1573}.


%\bigskip
%\noindent Une version officielle de Java 1.7 existe pour OS X 10.5, 64-bit Intel 
%(Core 2 Duo), mais il n'y a pas de solution officielle pour les anciens OS X (10.4 ou plus anciens),
%PowerPC et 32-bit Intel (Core Duo). Ainsi,
% si vous avez OS X 10.5, un OS X 64-bit Intel, il vous suffit de vous procurer
%	la JRE 1.7. Apple. Le seul problème est que cette version ne démarre pas par défaut.
%	Voir section ``Java pour Mac OS X 10.5 Update 10'', à \url{https://support.apple.com/kb/DL1359}


%\noindent\textbf{Comment savoir si mon processeur est un 32 ou un 64 bits ?}

%\noindent Dans le menu Apple, cliquez sur "About this Mac". Si vous voyez quelquechose comme:
%"Processor : x.xx Ghz Intel Core Duo", votre processeur est un 32 bits.

%\bigskip
%\noindent Si vous voyez "Processeur: x.xx Ghz Intel Core 2 Duo", ou si votre
%processeur est de type Intel (comme Xeon), alors vous avez un processeur 64 bits.

%\subsection{Utiliser Apple Java 1.7 runtime}
%\bigskip\index{Java!Apple Java 1.7 runtime}
%\noindent Si vous utilisez Mac OS X 10.5 (ou ultérieur) sur des processeurs Intel 64 bits, vous pouvez simplement utiliser le Java 1.7 %d'Apple. Vous pouvez l'obtenir à partir de \url{https://support.apple.com/kb/DL1359}.

%\noindent Vous pouvez aller dans Application -> Utilities -> Java Preferences pour vérifier la présence  de "Java SE 7" dans la liste %"Java Applications".


%\subsubsection{Option 1 : modifier le runtime par défaut pour Java Applications}
%\noindent Si vous n'utilisez pas une autre application Java qui a besoin de Java 1.6, vous pouvez
%simplement mettre "Java SE 7" en haut de la liste «Applications Java" dans Utilitaire de préférence
%Java.
%\subsubsection{Option 2 : Créer un alias pour lancer Java 1.7}
%\noindent Si vous ne voulez pas modifier les paramètres globaux de Java, vous pouvez créer un alias

%\bigskip
%\noindent \verb+alias jre7="/System/Library/Frameworks/JavaVM.framework/Versions/+
%\noindent \verb+1.7/Commands/java"+
   
%\bigskip
%\noindent \verb+jre7 -jar Unitex.jar+

%\bigskip
%\noindent Ensuite lancer Unitex depuis un terminal.

%\subsection{SoyLatte}

%\subsection{Comment compiler les programmes les C++ Unitex sur un ordinateur Macintosh}


%\subsection{Comment rendre tous les fichiers visibles sur Mac OS}
%\noindent Voir
%\url{http://www.macworld.com/article/51830/2006/07/showallfinder.html}.

%\bigskip
%\noindent Ou essayez tout de suite... Tapez: 

%\bigskip
%\verb+defaults write com.apple.Finder AppleShowAllFiles ON+

%\bigskip
%\noindent Ensuite redémarrez le Finder:

%\bigskip
%\verb+killall Finder+

%\begin{figure}[!ht]
%\begin{center}
%\includegraphics[width=12cm]{resources/img/fig-mac6.png}
%\caption{Redémarrez le Finder\label{fig-mac6}}
%\end{center}
%\end{figure}

%\bigskip
%\noindent Pour revenir à la configuration d'origine, tapez: 

%\bigskip
%\verb+defaults write com.apple.Finder AppleShowAllFiles OFF+


\section{Première utilisation}
\label{section-first-use}
Si vous travaillez sous Windows, le programme vous demandera de choisir un répertoire personnel de travail
\index{Répertoire!personnel de travail}, que vous pourrez changer ultérieurement dans
"Info>Preferences...>Di-rectories". Pour créer un répertoire, cliquez sur l’icône représentant un
dossier (voir figure~\ref{fig-creation-personal-directory}).

\bigskip
\noindent Sous Linux et OS X, le programme créera automatiquement un répertoire personnel de travail,
appelé \verb+/unitex+, dans votre répertoire \verb+$HOME+. 

\bigskip
\noindent Le répertoire personnel de travail, ou répertoire de l'utilisateur, vous permettra de stocker vos
données Unitex personnelles. Pour chaque langue que vous utiliserez, le
programme copiera l’arborescence de la langue dans votre répertoire de travail,
à l’exception des dictionnaires. Vous pourrez ainsi modifier à votre guise votre copie des données
sans risquer d’endommager les données du système, stockées dans le
répertoire système Unitex.\index{Répertoire!système Unitex}


\begin{figure}[!ht]
\begin{center}
\includegraphics[width=6.3cm]{resources/img/fig1-1.png}
\caption{Première utilisation sous Windows}
\end{center}
\end{figure}

\begin{figure}[!ht]
\begin{center}
\includegraphics[width=7cm]{resources/img/fig1-2.png}
\caption{Première utilisation sous Linux}
\end{center}
\end{figure}

\begin{figure}[!ht]
\begin{center}
\includegraphics[width=13cm]{resources/img/fig1-3.png}
\caption{Création du répertoire personnel de travail
\label{fig-creation-personal-directory}}
\end{center}
\end{figure}



\section{Ajout de nouvelles langues}
\index{Ajout de nouvelles langues}

\bigskip
\noindent Il y a deux manières d’ajouter des langues. Si vous désirez ajouter une nouvelle langue
accessible à tous les utilisateurs, il vous faut copier le répertoire correspondant à cette langue
dans le répertoire système Unitex,\index{Répertoire!système Unitex}
ce qui nécessite d’avoir les droits d’accès à ce répertoire
(il vous faudra peut-être demander à votre administrateur système de le faire).
En revanche, si vous êtes le seul utilisateur concerné par la langue, vous pouvez copier le répertoire
en question dans votre répertoire de travail.\index{Répertoire!personnel de travail}
Vous pourrez ainsi travailler sur cette langue sans qu’elle soit proposée aux autres utilisateurs.



\section{Désinstallation}
Quel que soit le système sous lequel vous travaillez, il vous suffit de supprimer le répertoire
\verb+Unitex+ pour effacer tous les fichiers du système. Sous Windows, vous devrez ensuite supprimer
le raccourci vers \verb+Unitex.jar+ \index{Fichier!\verbc{Unitex.jar}} si vous en avez créé un ;
même chose sous Linux ou OS X si vous avez créé un alias.


\section{Unitex pour les développeurs}
\label{section-unitex-developers}
Si vous êtes programmeur, cela peut vous intéresser de lier votre code avec les sources C++
d'Unitex. Pour faciliter cette opération, vous pouvez compiler Unitex en tant que bibliothèque
dynamique qui contient toutes les fonctions Unitex, sauf les \verb+main+s, bien sûr. La
page \url{http://docs.unitexgramlab.org/projects/unitex-library/fr/latest/} est une documentation sur la bibliothèque.
Les sources C++ d'Unitex contiennent du code pour des bindings Java JNI, Ruby et Microsoft .NET. La
page \url{https://github.com/patwat/python-unitex} contient des bindings Python.


\bigskip
Sous Linux/OS X, tapez~:

\bigskip
\verb+make LIBRARY=yes+

\bigskip
\noindent et vous obtiendrez une bibliothèque nommée \verb+libunitex.so+. Si vous souhaitez produire 
DLL Windows nommée \verb+unitex.dll+, utilisez les commandes suivantes:

\bigskip
Windows: \verb+make SYSTEM=windows LIBRARY=yes+

Cross-compilation avec mingw32: \verb+make SYSTEM=mingw32 LIBRARY=yes+

\bigskip
\noindent dans tous les cas, vous obtiendrez aussi un programme nommé
\verb+Test_lib+(\verb+.exe+). Si tout a bien fonctionné, ce programme devrait afficher l'écran
suivant:

\begin{verbatim}
Expression converted.
Reg2Grf exit code: 0

#Unigraph
SIZE 1313 950
FONT Times New Roman:  12
OFONT Times New Roman:B 12
BCOLOR 16777215
FCOLOR 0
ACOLOR 12632256
SCOLOR 16711680
CCOLOR 255
DBOXES y
DFRAME y
DDATE y
DFILE y
DDIR y
DRIG n
DRST n
FITS 100
PORIENT L
#
7
"<E>" 100 100 1 5
"" 100 100 0
"a" 100 100 1 6
"b" 100 100 1 4
"c" 100 100 1 6
"<E>" 100 100 2 2 3
"<E>" 100 100 1 1
\end{verbatim}
