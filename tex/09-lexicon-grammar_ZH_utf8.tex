\chapter{语法和词汇}
\label{chap-lexicon-grammar}
\index{Lexique-grammaire}
词典 - 语法表是表示语言的元素的句法属性的一个紧凑的方式。能够通过一个参数化机构自动构建从这些表中的本地文法,图表。

\bigskip
\noindent本章的第一部分介绍了这些表的形式主义。第二部分介绍了参数化图形和从语法词汇表生成图形自动机制。



\section{语法词汇表}
\index{Lexique-grammaire!table}\index{Matrices}
语法词汇已被Maurice Gross和他的团队开发了一种方法LADL\index{LADL} (\cite{L}, \cite{BGL}, \cite{methodes-en-syntaxe}, \cite{GL},
\cite{gross1994}, \cite{gross1994b}, \cite{gross1991}, \cite{gross1986}, 
\cite{gross1984}, \cite{gross1984b}, \cite{gross1983}, \cite{gross1982}, 
\cite{gross1978}, \cite{leclere2005}, \cite{salkoff2004})
主要原则是每个动词几乎单一的语法属性。因此,这些性能必须系统描述的,因为它是不可能预测动词的精确的行为。这些系统的描述是通过矩阵,其中的行对应的动词,和列语法属性表示。所考虑的特性是正式性能如由动词允许添加和所述不同的转换,这动词可经历(被动,名词,外位等)的数量和性质。矩阵通常也称为表,是二进制:一个符号\verb$+$如果动词检查属性,标志\verb+-+出现在一个行的交叉点和属性的列,否则用\index{Propriétés syntaxiques}。有关更多信息,语法属性看\url{http://infolingu.univ-mlv.fr},其中词汇,语法表都可以免费下载。

\bigskip
\noindent这种类型的描述的也用于形容词(\cite{these-annie}),表语(\cite{les-nominalisations},\cite{les-predicats-nominaux}, \cite{giry1978}, \cite{gross1976},
\cite{ranchhod2001}), adverbes (\cite{syntaxe-de-ladverbe},
\cite{grammaire-des-adverbes}), 在许多语言中有很多固定表达,
(\cite{lexique-grammaire-allemand2}, \cite{lexique-grammaire-italien2},
\cite{lexique-grammaire-italien}, \cite{lexique-grammaire-coreen2},
\cite{lexique-grammaire-coreen}, \cite{lexique-grammaire-malgache},
\cite{lexique-grammaire-espagnol}, \cite{lexique-grammaire-allemand},
\cite{lexique-grammaire-hongrois}, \cite{ranchhod1996}, \cite{ranchhod1991},
\cite{gross1986b}).

\bigskip
\noindent 图~\ref{fig-table-32NM}显示语法词汇的列表。此表涉及动词承认数字的补充。


\begin{figure}[!h]
\begin{center}
\includegraphics[width=15cm]{resources/img/fig8-1.png}
\caption{32NM语法表\label{fig-table-32NM}}
\end{center}
\end{figure}

\section{转换图表}
\subsection{图表主要参数}
\index{Graphe!paramétré}
变换的图表是由图机构的装置
设置。其原理是:我们建立一个描述可能的结构图。此图通过变量引用表列。然后对表中,该曲线图,其中,变量根据位于对应的列的交叉点,并经处理的线的单元中的内容置换的副本中的每一行生成。如果表格单元格中包含的标志\verb$+$相应的变量由\verb+>+取代。如果单元格包含符号\verb+-+,含有相应变量的框被删除,同时删除这个框的路径。在所有其他情况下,该变量修改为被替换的单元的内容。


\subsection{表的格式}
词汇,语法表使用电子表格程序通常编码比如OpenOffice.org Calc (\cite{OpenOffice}).要使用Unitex,该表必须在Unicode文本使用以下约定编码:列必须在标签和行由回车符分隔。

\bigskip
\noindent要转换使用OpenOffice.org Calc的表,保存文本文件(拓展名为\verb$.csv$)。然后该程序通过提供类似图~\ref{fig-csv-export}.窗口设置备份。选择编码“Unicode”,选择制表符分隔列,不指定文字分隔符。

\begin{figure}[!ht]
\begin{center}
\includegraphics[width=12cm]{resources/img/fig8-2.png}
\caption{使用OpenOffice.org Calc的备份配置表\label{fig-csv-export}}
\end{center}
\end{figure}

\bigskip
\noindent当生成图表,Unitex跳过第一行,考虑给予列的标题。因此,您必须确保列的标题正好占据一行。如果没有标题行,表中的第一行会被忽略,并且有几个头线,它们将来自第二线条被解释表。



\subsection{图的参数}
参数化曲线是显示在参照表语法词汇的列中的变量的曲线图。通常使用这种机制与语法图,但没有什么会阻止建立参数化图形,或预处理标准。

\index{Variable!dans un graphe paramétré}
\bigskip
\noindent该参考列中的变量字符的形成\verb+@+\index{\verb+"@+}其次是用大写字母列名(列的编号从\verb+A+).

\bigskip
\noindent 例子: \verb+@C+参照该表的第三列。

\bigskip
\noindent当变量是由\verb$+$ 或 \verb+-+代替, \verb+-+标记对应于通过这个变量去除的方式。
有可能通过字符前述执行反向操作 \verb+@+一个感叹号。在这种情况下,这是当变量是指标志\verb$+$路径被删除。如果变量返回或标志既不是\verb-+-,也不是 \verb+-+,它是由格子中的替换内容。

\bigskip
\noindent还有一个特殊的变量\verb+@%+ \index{\verb+"@%+},这是由在表中的行号代替。它的值是每行不同的事实允许使用容易地表征的线。这个变量不会受到感叹号的在左边。


\bigskip
\noindent 图~\ref{fig-reference-graph}给出了示例参数化的曲线图,旨在被应用到如图~\ref{fig-table-31H}词库的无关文法表中的31H表。

\begin{figure}[!h]
\begin{center}
\includegraphics[width=15cm]{resources/img/fig8-3.png}
\caption{图的参数的实例\label{fig-reference-graph}}
\end{center}
\end{figure}

\begin{figure}[!h]
\begin{center}
\includegraphics[width=15cm]{resources/img/fig8-4.png}
\caption{词汇语法31H表\label{fig-table-31H}}
\end{center}
\end{figure}


\subsection{自动生成图像}
从参数化图形和表格生成图表,它必须首先通过点击菜单中的“词汇,语法”打开表“打开...”(参见~\ref{fig-lexicon-grammar-menu})。该表必须已转换为Unicode文本。
\begin{figure}[!h]
\begin{center}
\includegraphics[width=12cm]{resources/img/fig8-5.png}
\caption{Menu "Lexicon-Grammar"\label{fig-lexicon-grammar-menu}}
\end{center}
\end{figure}

\bigskip
\noindent(见图~\ref{fig-table-display})选定的表格显示在一个窗口。如果它不显示在屏幕上,它可以被其他窗口Unitex被隐藏。

\begin{figure}[!h]
\begin{center}
\includegraphics[width=15cm]{resources/img/fig8-6.png}
\caption{Displaying a table\label{fig-table-display}}
\end{center}
\end{figure}

\bigskip
\noindent  要自动生成一个参数化图形图表,请点击“编译到GRF...”,从“词汇,语法”菜单。出现如图\ref{fig-configuration-graph-generation}。

\begin{figure}[!h]
\begin{center}
\includegraphics[width=9cm]{resources/img/fig8-7.png}
\caption{确认自动生成图像\label{fig-configuration-graph-generation}}
\end{center}
\end{figure}

\bigskip
\noindent在“参考图形(在GRF格式)”,然后设置为使用图的名称。在“结果GRF语法”中,指定将要生成的主图的名称。主图表是利用已生成的所有图的曲线图。通过搜索这个图形文字,你会同时应用和所有生成的图表。


\bigskip
\noindent  设置“子图名”让你指定要生成的图的名称。可以肯定,所有的图形都会有不同的名称,建议使用变量\verb+@+,这个变量将被替换为通过它的数字每个输入,以确保所有的图表都不同的名称。例如,如果填充的名称帧"\verb+TestGraph.grf+",并且如果子图被命名为"\verb+TestGraph_@.grf+",从16产生的子图第16线将被命名为"\verb+TestGraph_0016.grf+"。

\bigskip
\noindent 图\ref{fig-archaiser} 和图\ref{fig-badauder}表明通过施加图的参数化曲线产生两个图~\ref{fig-reference-graph}在表31H.

\bigskip
\noindent 图~\ref{fig-main-graph}显示获得的主要图。

\begin{figure}[!h]
\begin{center}
\includegraphics[width=15cm]{resources/img/fig8-8.png}
\caption{对于动词生成的图表
\texttt{archaiser}\label{fig-archaiser}}
\end{center}
\end{figure}

\begin{figure}[!h]
\begin{center}
\includegraphics[width=15cm]{resources/img/fig8-9.png}
\caption{对于动词生成的图表 \texttt{badauder}\label{fig-badauder}}
\end{center}
\end{figure}

\begin{figure}[!h]
\begin{center}
\includegraphics[width=10cm]{resources/img/fig8-10.png}
\caption{主图调用所有生成的图\label{fig-main-graph}}
\end{center}
\end{figure}



