\chapter{安装Unitex}
\label{chap-install}
Unitex\ 是一个多平台软件,它能在\ Windows、Linux 或\ MacOS 下良好运行。这一章节描述了\ Unitex 在这些操作系统下的安装和运行。这一章节也阐述了加入新语言和卸载的方法。

\section{证书}
\label{section-licences}
\index{LGPL 证书}\index{LGPL 证书}
Unitex 是一个开放的软件,也就是说这些工程的资源是由该软件分配的,而且每个人都能修改或再分配。Unitex 工程的代码是遵循\ LPGL 证书的(\cite{LGPL}),除了对常规表达的操作库\ TRE(\cite{TRE}),这遵循着证书\ "2 Clause BSD" (比\ LPGL 证书更加宽松),而\ LibYAML 库遵循的\ MIT 证书也比\ LGPL 更宽松。
\ LGPL 证书比\ GPL 宽松,因为它能在不开源的软件中使用\ LGPL 代码。从用户们的角度两者没有区别,因为他们都能自由的使用和分配。

\bigskip
\noindent所有\ Unitex 分配的语言学暑假都遵循\ LGPLLR 证书
\index{证书\ LGPLLR} (\cite{LGPLLR}).

\bigskip
\noindent GPL,LGPL 和\ LGPLLR 证书的完整文本都在用户手册的附件中。

\section{执行\ Java 的环境}
Unitex 包含了由\ Java 语言编写的图形界面而外部程序由 \textit{C/C\kern-.05em\raisebox{.5ex}{++}\kern-.1em}。这种编程语言的混合使用使程序更迅速并能在多个操作系统下运行。


\bigskip
\noindent 为了使用图形界面,需要先准备一个虚拟机\index{Java虚拟机}或\ 
JRE\index{JRE} (Java 运行环境\index{Java 运行环境}\index{Java!JRE}).

\bigskip
\noindent 在图形模式下运行,Unitex 需要1.6(或更高版本)的\ Java 版本。如果你还是\ Java 的旧版本,Unitex 将你选择了你的工作语言后锁定。

\bigskip
\noindent 您可以随意下载的虚拟机操作系统从\ Sun Microsystems 的网站 (\cite{site-java}) : 
\url{http://java.sun.com}.

\bigskip
\noindent 如果你正在使用\ Linux 或\ MacOS 或如果你使用一个版本的\ Windows 的管理做实个人账户的用户,你需要问你的系统管理员安装\ Java。



\section{Windows 下安装}
\index{Installation!sous Windows}
下载文件的名字如下:
\begin{flushleft}
{\tt \UnitexPackageWin{}}
{\tt \UnitexPackageWinSF{}}
\end{flushleft}




\bigskip
\noindent 解压缩文件 \index{文件!\verb+Unitex3.1beta.zip+} \verb+Unitex3.1beta.zip+ (或 \verb+Unitex3.0.zip+)
--- 您可以在以下地址下载这些文件: \url{http://igm.univ-mlv.fr/~unitex} ---
在提前创建的\ \verb+Unitex3.1beta+ 目录(文件夹)中,
在偏好目录\ \verb+Program Files+,而它会调用系统目录中的\ Unitex。\index{系统目录\ Unitex}\index{文件夹|见{目录}}

\bigskip
\noindent 解压缩之后,目录 \ \verb+Unitex3.1beta+
(le répertoire système Unitex)  包含了多个名为\ \verb+App+ 的子目录。 最后一个目录包含了名为\ \verb+Unitex.jar+ \index{文件!\verb+Unitex.jar+}的文件。
 该\ Java 文件用于运行图形界面。您仔细双击该图表就能运行该软件。
为了更方便地运行,建议您在桌面建立快捷方式。

\section{Linux 安装}
\index{Linux 环境下的安装}
在\ Linux 和\ MacOS 系统上安装\ Unitex,要求管理员权限。您将文件\ \verb+Unitex3.1beta.zip+ 解压缩于名为\ \verb+Unitex+ 的文件夹,通过以下命令:


\bigskip \noindent \verb$unzip Unitex3.1beta.zip -d Unitex$

\bigskip
\noindent 该目录将会调用系统目录\ Unitex\index{系统目\ 录Unitex},
接着您在目录 \ \verb|Unitex/Src/C++/build| 中通过以下命令进行编译:


\bigskip \verb+安装+

\bigskip
\noindent 如果是六十四位计算机:
 
\bigskip \verb+make install 64BITS=yes+
\bigskip
\noindent 然后在接下来的模型中创建一个别名在:

\bigskip \verb$alias unitex='cd /..../Unitex/App/ ; java -jar Unitex.jar'$


\section{MacOS X 系统安装\ Unitex}
\index{MacOS X 系统安装\ Unitex}
\label{section-macos-install}
\noindent 注:这个简短的教程将告诉你如何在\ Mac OS X 系统中 安装和运行\ Unitex。
欢迎提出你的问题或对我们的评价和建议。
\noindent 请联系: \url{cedrick.fairon@uclouvain.be}


\bigskip
\noindent Java 的 \ Oracle 正式版 适用于 \ MacOS X 10.7.3 (Lion) 至最新版本。
	查看\ Mac OS X 系统安装和使用\ Oracle Java 的基本配置及详细说明'' 于以下网页\  \url{https://www.java.com/fr/download/faq/java_mac.xml}

	

\bigskip
\noindent 适用于 \ MacOS X 10.7 至最新版本 的 \ Apple 的\ Java 结构,
	详情见以下网页 \url{https://support.apple.com/kb/DL1572}. 
	适用于\  MacOS X 10.6 的 \ Apple 的结构,
	详情见以下网页\  \url{https://support.apple.com/kb/DL1573}.

\bigskip
\noindent Java 1.6 正式版 适用于\ MacOS X 10.5, 64-bit Intel (Core 2 Duo), 但对于\ OS X 的更早版本(10.4 或更早)没有正式的解决方法,
PowerPC 和\ 32-bit Intel (Core Duo)。然而,如果你的系统是\ OS X 10.5( MacOS 64-bit Intel),只需有 \ JRE 1.6. Apple。唯一的问题是这个版本不能默认启动。
	查看 \ ``Java 适用于 \ Mac OS X 10.5 Update 10'', 于以下网页 \ \url{https://support.apple.com/kb/DL1359}


\noindent\textbf{如何查看处理器是三十二位还是六十四位?}

\noindent Apple菜单中, 点击 "关于本机"。你将看到以下条目:
\ "Processor : x.xx Ghz Intel Core Duo", 您的处理器是\ 32 bits。

\bigskip
\noindent 如果你看到:"Processeur: x.xx Ghz Intel Core 2 Duo", 或你的处理器是\ Intel(如Xeon) 的, 那么你的处理器是六十四位的。

\subsection{使用 \ Apple Java 1.6 runtime}
\bigskip\index{\ Java!Apple Java 1.6 runtime}
\noindent 如果你使用 \ Mac OS X 10.5 (或更高版本) 于\ Intel 64 bits 处理器, 你就能轻松地使用\ Java 1.6 d'Apple。
详细说明于以下网页\  \url{https://support.apple.com/kb/DL1359}.

\noindent 点击进入 \ Application -> Utilities -> Java Preferences 
 然后在\ "Java Applications" 列表中 找到\ "Java SE 6"。

\subsubsection{选项\ 1:更改\ Java 应用程序的默认运行}
\noindent 如果你不使用\ Java 1.5 应用程序,
那么你可以按照个人使用习惯,在\ "Applications Java" 列表中将\ "Java SE 6" 置顶。

\subsubsection{选项\ 2:创建一个别名运行\ Java1.6}
\noindent 如果你不想修改\ Java 中的全局变量,你能创建一个别名。

\bigskip
\noindent \verb+alias jre6="/System/Library/Frameworks/JavaVM.framework/Versions/+
\noindent \verb+1.6/Commands/java"+
   
\bigskip
\noindent \verb+jre6 -jar Unitex.jar+

\bigskip
\noindent 然后从终端运行\ Unitex

%\subsection{SoyLatte}

%\subsection{Comment compiler les programmes les C++ Unitex sur un ordinateur Macintosh}


\subsection{如何使\ Mac OS 中的所有文件可见}
\noindent 查看
\url{http://www.macworld.com/article/51830/2006/07/showallfinder.html}.

\bigskip
\noindent 或立即尝试... 输入: 

\bigskip
\verb+defaults write com.apple.Finder AppleShowAllFiles ON+

\bigskip
\noindent 然后重新运行\ Finder:

\bigskip
\verb+killall Finder+

\begin{figure}[!h]
\begin{center}
\includegraphics[width=12cm]{resources/img/fig-mac6.png}
\caption{重新运行 \ Finder\label{fig-mac6}}
\end{center}
\end{figure}

\bigskip
\noindent 要返回原来的配置,需要输入:

\bigskip
\verb+defaults write com.apple.Finder AppleShowAllFiles OFF+


\section{第一次使用}
如果您使用的是\ Windows,你需要选择一个工作主目录
\ \index{Répertoire!personnel de travail}。 您可以在“信息>首选项...>目录”更改。要创建一个文件夹,点击文件夹图标
(见图 \ref{fig-creation-personal-directory}).

\bigskip
\noindent 如果您使用的是\ Linux 和\ Mac OS,该程序会自动创建一个个人工作目录,称为\ \verb+/unitex+ 在目录\verb+$HOME+. 

\bigskip
\noindent 在个人工作目录,您将存储您的\ Unitex 个人资料。您所使用的每种语言,程序会复制语言的树在你的工作目录中,除了字典。然后,您可以编辑,你需要不损坏的复制你的数据存放在\ Unitex 系统目录系统数据。\index{Répertoire!système Unitex}


\begin{figure}[h]
\begin{center}
\includegraphics[width=6.3cm]{resources/img/fig1-1.png}
\caption{\ Windows 系统第一次使用}
\end{center}
\end{figure}

\begin{figure}[h]
\begin{center}
\includegraphics[width=7cm]{resources/img/fig1-2.png}
\caption{\ Linux 系统第一次使用}
\end{center}
\end{figure}

\begin{figure}[h]
\begin{center}
\includegraphics[width=13cm]{resources/img/fig1-3.png}
\caption{个人工作目录的建立
\label{fig-creation-personal-directory}}
\end{center}
\end{figure}



\section{添加新的语言}
\index{Ajout de nouvelles langues}

\bigskip
\noindent 有两种方法来添加语言。如果你想添加一个新的语言到所有用户,你必须复制进\ Unitex 系统语言目录,\index{Répertoire!système Unitex}
这需要有访问该目录的权利(你可能需要询问系统管理员).但是,如果是这语言有关的唯一用户,你可以在你的工作目录复制目录。\index{Répertoire!personnel de travail} 您将能够在这语言工作而不提供给其他用户。


\section{卸载}
不管是什么系统,你的工作,你只需删除\ \verb+Unitex+ 目录所有的系统文件。在\ Windows 上,则必须删除快捷方式\ \verb+Unitex.jar+ \index{Fichier!\verb+Unitex.jar+}  如果您已创建了一个;同样在\ Linux 或\ MacOS 上,如果你已经创建了一个别名。


\section{Unitex 开发}
\label{section-unitex-developpers}

如果你是一个程序员,那你可能感兴趣是\ Unitex 的\ C++ 代码来源的链接。为了推动这项工作,你就可以编译\ Unitex 为包含除库\ \verb+main+s 所有功能以外\ Unitex 功
能的动态。\url{http://docs.unitexgramlab.org/projects/unitex-library/fr/latest/}页面包含有关库文档。


\bigskip
Linux/MacOS 下, 输入:

\bigskip
\verb+make LIBRARY=yes+

\bigskip
\noindent 你会得到一个库命名为 \ \verb+libunitex.so+. 如果你想制作的\ Windows DLL 文件名为\ \verb+unitex.dll+,请使用以下命令:

\bigskip
Windows: \verb+make SYSTEM=windows LIBRARY=yes+

带\ mingw32 的交叉编译: \verb+make SYSTEM=mingw32 LIBRARY=yes+

\bigskip
\noindent 在所有情况下,你也将获得一个所谓的\ AA 程序\ 
\verb+Test_lib+(\verb+.exe+). 
如果一切行之有效,该程序应显示以下画面:

\begin{verbatim}
Expression converted.
Reg2Grf exit code: 0

#Unigraph
SIZE 1313 950
FONT Times New Roman:  12
OFONT Times New Roman:B 12
BCOLOR 16777215
FCOLOR 0
ACOLOR 12632256
SCOLOR 16711680
CCOLOR 255
DBOXES y
DFRAME y
DDATE y
DFILE y
DDIR y
DRIG n
DRST n
FITS 100
PORIENT L
#
7
"<E>" 100 100 1 5
"" 100 100 0
"a" 100 100 1 6
"b" 100 100 1 4
"c" 100 100 1 6
"<E>" 100 100 2 2 3
"<E>" 100 100 1 1
\end{verbatim}
